\documentclass{resume}
\usepackage{zh_CN-Adobefonts_external} 
\usepackage{linespacing_fix}
\usepackage{cite}
\usepackage{hyperref}
\usepackage{multicol}
% 精致项目符号(纯格式)
\renewcommand{\labelitemi}{$\vcenter{\hbox{\tiny$\bullet$}}$}

\begin{document}
\pagenumbering{gobble}

\name{孙楚芮}
\contactInfo{(+86) 15204503697}{crsun@ir.hit.edu.cn}
\otherInfo{性别:男}{籍贯:黑龙江}{}{}
\yourphoto{0.14}
\vspace{-8mm}
% ==================== 教育背景 ====================
\section{教育背景}

\datedsubsection{\textbf{哈尔滨工业大学},计算机科学与技术,\textit{学士}}{2020.09 -- 2024.06}
\datedsubsection{\textbf{哈尔滨工业大学},电子信息(人工智能方向),\textit{硕士}}{2024.09 -- 至今}


% ==================== 科研经历 ====================
\section{科研经历}

\datedsubsection{\textbf{哈尔滨工业大学 SCIR 实验室},实习生}{2022.10 -- 2024.06}
\begin{itemize}[parsep=0.8ex]
    \item \textbf{\href{https://aclanthology.org/2023.findings-acl.453/}{多语言明喻对话数据集构建与模型评测}}  
  参与构建中英双语明喻对话数据集,包含近 2 万条人工标注样本;设计评测方案分析主流模型在明喻理解与生成上的表现。成果发表于 \textbf{ACL Findings 2024}(CCF-A类会议)。
  
  \item \textbf{\href{https://aclanthology.org/2023.dstc-1.24/}{融合外部知识的任务型对话系统}}  
在 DSTC-11 主观知识对话赛道中提出差异感知集成方法,融合多模型预测以提升知识选择鲁棒性;负责集成模块实现与生成模型微调,助力团队获知识选择单项第一名、总成绩第三名。成果发表于 \textbf{SigDial 2023}(ACL Workshop)。
  
\item \textbf{\href{https://aclanthology.org/2024.lrec-main.138/}{基于自验证机制的明喻知识挖掘}}  
  参与构建多层级明喻质量数据集,支持从预训练语言模型中自动挖掘高置信度明喻知识。成果发表于 \textbf{COLING 2024}(CCF-B 类会议)。
\end{itemize}

\datedsubsection{\textbf{哈尔滨工业大学 SCIR 实验室},硕士研究生}{2024.09 -- 至今}
\begin{itemize}[parsep=0.8ex]
\item \textbf{\href{https://doi.org/10.1002/smb2.70003}{具身智能综述:从感知到行为智能}}  
  参与撰写关于具身人工智能的系统性综述,负责“场景感知”章节,梳理视觉-语言融合、环境建模与主动感知等方向的研究进展。文章发表于 \textbf{SmartBot 2025}
  
\item \textbf{\href{https://arxiv.org/abs/2512.08548}{基于语言抽象的机器人动作表征学习}}  
  设计分层自适应动作识别方法,将连续轨迹转化为结构化自然语言动作指令,有效弥合高层指令与底层控制间的尺度差异;参与模型训练、跨任务泛化实验及论文撰写。成果发表于 \textbf{AAAI 2026}(CCF-A 类会议)。
  
\item \textbf{外部知识驱动的类比推理任务与评测}  
  参与构建新型类比推理数据集,负责多类型自动评测方案设计及主流大语言模型测试;部分参与论文撰写。成果已投稿至 \textbf{ACL 2026}(CCF-A 类会议)。
\end{itemize}

% ==================== 项目经历 ====================
\section{项目经历}

\datedsubsection{\textbf{小红导览机器人:基于多模态知识库的智能问答系统}}{2024.04 -- 2024.06}
\begin{itemize}[parsep=0.8ex]
  \item 面向实时游客问答场景构建混合检索架构:在非 GPU 环境下采用分词+关键词匹配(自研规则引擎),GPU 环境下基于 \textbf{BGE} 向量模型进行语义检索,并引入 \textbf{Qwen-Reranker} 对召回结果重排序;
  \item 集成对话历史感知的问题重写机制与父子文档索引策略,提升上下文相关问题的检索精度;
  \item 系统上线哈工大中心,单日最高服务游客 1.2 万人次;配套宣传视频获 1000+ 点赞、890 次转发。
\end{itemize}

\datedsubsection{\textbf{机器脑:多模态人机协同执行系统}}{2024.11 -- 2024.12}
\begin{itemize}[parsep=0.8ex]
  \item 负责本地化大语言模型部署与推理优化,通过量化、边缘化部署、并行化与 prompt 工程,将首词生成延迟从 4.0s 降至 1.4s,显著提升交互流畅性;
  \item 构建“语言-视觉-动作”闭环执行 pipeline:用户指令经 LLM 解析后,调用 \textbf{Grounding DINO} 实现开放词汇物体定位,拼接点云数据输入 \textbf{AnyGrasp} 生成抓取位姿,驱动机械臂完成操作;
  \item 在桌面物品整理、汉诺塔求解等场景中验证能力,整体成功率超 \textbf{90\%},支撑多次校内外参观演示。
\end{itemize}

\datedsubsection{\textbf{面向推理短板的合成数据构造系统}}{2025.07 -- 至今}
\begin{itemize}[parsep=0.8ex]
  \item 基于模型能力诊断(如推理弱项,知识短板),自动构造针对性训练样本;
  \item 引入对抗性生成与辅助回答机制,确保合成数据质量与知识覆盖度。
\end{itemize}

% ==================== 技术能力(仅格式:双栏 + 大小写规范)====================
\section{技术能力}
\begin{multicols}{2}
\begin{itemize}[parsep=0.5ex,left=0pt]
  \item 编程语言:Python(熟练)、C++、C、Java、shell
  \item 框架:PyTorch, Tensorflow, Transformers, Ros2
  \item 工具:Mujoco, LoRA, vLLM, Git, Latex
  \item 语言能力:英语(CET-6 520+),可流畅阅读论文
\end{itemize}
\end{multicols}

\end{document}